% !TEX TS-program = pdflatex
% !TEX encoding = UTF-8 Unicode

% This is a simple template for a LaTeX document using the "article" class.
% See "book", "report", "letter" for other types of document.

\documentclass[11pt]{article} % use larger type; default would be 10pt

\usepackage[utf8x]{inputenc}
\usepackage[T1]{fontenc}
\usepackage[polish]{babel}
\usepackage{hyperref}



%%% Examples of Article customizations
% These packages are optional, depending whether you want the features they provide.
% See the LaTeX Companion or other references for full information.

%%% PAGE DIMENSIONS
\usepackage{geometry} % to change the page dimensions
\geometry{a4paper} % or letterpaper (US) or a5paper or....
% \geometry{margin=2in} % for example, change the margins to 2 inches all round
% \geometry{landscape} % set up the page for landscape
%   read geometry.pdf for detailed page layout information

\usepackage{graphicx} % support the \includegraphics command and options

% \usepackage[parfill]{parskip} % Activate to begin paragraphs with an empty line rather than an indent

%%% PACKAGES
\usepackage{booktabs} % for much better looking tables
\usepackage{array} % for better arrays (eg matrices) in maths
\usepackage{paralist} % very flexible & customisable lists (eg. enumerate/itemize, etc.)
\usepackage{verbatim} % adds environment for commenting out blocks of text & for better verbatim
\usepackage{subfig} % make it possible to include more than one captioned figure/table in a single float
% These packages are all incorporated in the memoir class to one degree or another...

%%% HEADERS & FOOTERS
\usepackage{fancyhdr} % This should be set AFTER setting up the page geometry
\pagestyle{fancy} % options: empty , plain , fancy
\renewcommand{\headrulewidth}{0pt} % customise the layout...
\lhead{}\chead{}\rhead{}
\lfoot{}\cfoot{\thepage}\rfoot{}

%%% SECTION TITLE APPEARANCE
\usepackage{sectsty}
\allsectionsfont{\sffamily\mdseries\upshape} % (See the fntguide.pdf for font help)
% (This matches ConTeXt defaults)

%%% ToC (table of contents) APPEARANCE
\usepackage[nottoc,notlof,notlot]{tocbibind} % Put the bibliography in the ToC
\usepackage[titles,subfigure]{tocloft} % Alter the style of the Table of Contents
\renewcommand{\cftsecfont}{\rmfamily\mdseries\upshape}
\renewcommand{\cftsecpagefont}{\rmfamily\mdseries\upshape} % No bold!

%%% END Article customizations

%%% The "real" document content comes below...

\title{MNUM Projekt, zadanie 1.10}
\author{Monika Pawluczuk, nr indeksu 246428}
%\date{} % Activate to display a given date or no date (if empty),
         % otherwise the current date is printed 

\begin{document}
\maketitle

\section{Zadanie 1}

\subsection{Treść polecenia}

Proszę napisać program wyznaczający dokładność maszynową komputera i wyznaczyć ją na swoim komputerze.

\subsection{Zastosowane algorytmy}

Dokładność maszynowa to maksymalny błąd względny reprezentacji zmiennoprzecinkowej równy 2 do -t, gdzie t to liczba bitów mantysy - a więc zależny wyłącznie jej liczby. Zgodnie z przyjętą konwencją, będę go oznaczać jako eps.
Równoważną definicją jest również najmniejsza dodatnia liczba maszynowa g taka, że zachodzi relacja 
\begin{equation}
fl(1 + g) > 1, tzn. eps = min \{  g \in M : fl(1 + g) > 1, g > 0\}
\end{equation}
Zgodnie z drugą definicją, przyjmę początkowy eps jako 1. Będę zmniejszać eps o połowę w każdej iteracji, dopóki eps + 1 > 1 (czyli eps > 0). Wyjście z pętli będzie oznaczało, że znaleźliśmy najmniejszy możliwy błąd.

\subsection{Implementacja użytych algorytmów}

\begin{verbatim}
function [ t, eps ] = machinePrecision()
%MACHINEPRECISION Return computer's machine precision
%   Return the least numeric value that is threated by the computer as
%   value above 0.
    eps = 1.0;
    t = 0;
    while (1.0 + eps/2.0 > 1.0)
        eps = eps/2.0;
        t = t + 1;
    end
    [t, eps];
end

\end{verbatim}

\subsection{Otrzymane wyniki oraz komentarz}

\begin{verbatim}
>> [t, eps] = machinePrecision()
t =
    52
eps =
   2.2204e-16
>> eps + 1.0 > 1.0
ans =
     1
\end{verbatim}
A więc liczba bitów mantysy to 52, natomiast dokładnośc maszynowa wynosi 2.2204e-16. Jest to wynik zgodny ze standardem IEEE 754 dla liczb zmiennoprzecinkowych podwójnej precyzji.

\section{Zadanie 2}

\subsection{Treść polecenia}

Proszę napisać program rozwiązując układ n równań liniowych Ax=b wykorzystując podaną metodę. Proszę zastosować program do rozwiązania podanych niżej układów równań dla rosnącej liczby równań n = 10, 20, 40, 80, 160, …. Liczbę tych równań proszę zwiększać do momentu, gdy czas potrzebny na rozwiązanie układu staje się zbyt duży/metoda zawodzi.
Metoda: Eliminacja Gaussa z częściowym wyborem elementu podstawowego
Dane:
1), 2), 3)
Dla każdego rozwiązania proszę obliczyć błąd rozwiązania (liczony jako norma residuum) i dla każdego układu równań proszę wykonać rysunek zależności tego błędu od liczby równań n.

\subsection{Zastosowane algorytmy}
Algorytm eliminacji Gaussa dzieli się na dwa etapy:

1. Eliminacja zmiennych - w wyniku przekształceń macierzy A i wektora b otrzymamy równoważny układ równań z macierzą trójkątną górną.

2. Postępowanie odwrotne (ang. back-substitution) - stosujemy algorytm rozwiązania układu z macierzą trójkątną.
\subsection{Implementacja użytych algorytmów}

\subsection{Otrzymane wyniki oraz komentarz}

\section{Zadanie 3}

\subsection{Treść polecenia}

Proszę napisać program rozwiązujący układ n równań liniowych Ax=b wykorzystując metodę Jacobiego i użyć go do rozwiązania poniższego układu równań liniowych:
Proszę sprawdzić dokładność rozwiązania oraz spróbować zastosować zaprogramowaną metodę do rozwiązania układów równań z zadania 2.

\subsection{Zastosowane algorytmy}

\subsection{Implementacja użytych algorytmów}

\subsection{Otrzymane wyniki oraz komentarz}

\end{document}
