% !TEX TS-program = pdflatex
% !TEX encoding = UTF-8 Unicode

% This is a simple template for a LaTeX document using the "article" class.
% See "book", "report", "letter" for other types of document.

\documentclass[11pt]{article} % use larger type; default would be 10pt

\usepackage[utf8]{inputenc} % set input encoding (not needed with XeLaTeX)
\usepackage[T1]{fontenc}
\usepackage[polish]{babel}
\usepackage{hyperref}
\usepackage{graphicx} 

%%% Examples of Article customizations
% These packages are optional, depending whether you want the features they provide.
% See the LaTeX Companion or other references for full information.

%%% PAGE DIMENSIONS
\usepackage{geometry} % to change the page dimensions
\geometry{a4paper} % or letterpaper (US) or a5paper or....
% \geometry{margin=2in} % for example, change the margins to 2 inches all round
% \geometry{landscape} % set up the page for landscape
%   read geometry.pdf for detailed page layout information

\usepackage{graphicx} % support the \includegraphics command and options

% \usepackage[parfill]{parskip} % Activate to begin paragraphs with an empty line rather than an indent

%%% PACKAGES
\usepackage{booktabs} % for much better looking tables
\usepackage{array} % for better arrays (eg matrices) in maths
\usepackage{paralist} % very flexible & customisable lists (eg. enumerate/itemize, etc.)
\usepackage{verbatim} % adds environment for commenting out blocks of text & for better verbatim
\usepackage{subfig} % make it possible to include more than one captioned figure/table in a single float
% These packages are all incorporated in the memoir class to one degree or another...

%%% HEADERS & FOOTERS
\usepackage{fancyhdr} % This should be set AFTER setting up the page geometry
\pagestyle{fancy} % options: empty , plain , fancy
\renewcommand{\headrulewidth}{0pt} % customise the layout...
\lhead{}\chead{}\rhead{}
\lfoot{}\cfoot{\thepage}\rfoot{}

%%% SECTION TITLE APPEARANCE
\usepackage{sectsty}
\allsectionsfont{\sffamily\mdseries\upshape} % (See the fntguide.pdf for font help)
% (This matches ConTeXt defaults)

%%% ToC (table of contents) APPEARANCE
\usepackage[nottoc,notlof,notlot]{tocbibind} % Put the bibliography in the ToC
\usepackage[titles,subfigure]{tocloft} % Alter the style of the Table of Contents
\renewcommand{\cftsecfont}{\rmfamily\mdseries\upshape}
\renewcommand{\cftsecpagefont}{\rmfamily\mdseries\upshape} % No bold!

%%% END Article customizations

%%% The "real" document content comes below...

\title{MNUM Projekt 2 - zadanie 2.19}
\author{Monika Pawluczuk, nr albumu 246428}
%\date{} % Activate to display a given date or no date (if empty),
         % otherwise the current date is printed 

\begin{document}
\maketitle

\section{Zadanie 1}

\subsection{Treść polecenia}

Proszę napisać program służacy do obliczania wartości własnych metodą rozkładu QR w dwóch wersjach: bez przesunięć i z przesunięciami dla macierzy symetrycznej, oraz w wersji z przesunięciami dla macierzy niesymatrycznej. Następnie proszę przetestować skuteczność (zbieżność) obu wersji algorytmu dla 30 różnych macierzy losowych o wymiarach: 5x5, 10x10, 20x20. Proszę podać średnią liczbę iteracji dla metody bez przesunięć i z przesunięciami. Dla wybranych macierzy proszę porównać otrzymane wyniki z wartościami własnymi obliczonymi poleceniem eig.

\subsection{Zastosowane algorytmy}

\subsection{Implementacja użytych algorytmów}

\subsection{Otrzymane wyniki i komentarz}

\section{Zadanie 2}

\subsection{Treść polecenia}

Dla następujących danych pomiarowych (próbek) metodą najmniejszych kwadratów należy wyznaczyć funkcję wielomianową y = f(x) najlepiej aproksymującą te dane (proszę przetestować wielomiany różnych rzędów). W sprawozdaniu proszę przedstawić na rysunku otrzymaną funkcję na tle danych. Do rozwiązania zadania najmniejszych kwadratów prosze wykorzystać:

a) układ równań normalnych,

b) układ równań liniowych z macierzą R wynikającą z rozkładu QR macierzy układu równań problemu.

Dla każdego układu równań proszę obliczyć błąd rozwiązania jako normę residuum (wektor residuum r = Ax - b).

\subsection{Zastosowane algorytmy}

\subsection{Implementacja użytych algorytmów}

\subsection{Otrzymane wyniki i komentarz}

\end{document}


